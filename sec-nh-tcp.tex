\section{Net Helper - TCP}
A first layout of the \textit{net-helper-TCP}'s module, that consist on initialization, main data structures
and management functions, were explained in the previous section. whereas its internals, that consist on
logical management of the data transfer, are explained in the follow one.

\subsection{Send to Peer}

\begin{lstlisting}
typedef enum {
    SENDER_IDLE,
    SENDER_BUSY
} sender_state_t;
\end{lstlisting}

\subsubsection{Sender - Interface}

\begin{lstlisting}
sender_t sender_new ();

sender_state_t sender_state (sender_t s);

void sender_reset (sender_t s);

int sender_subscribe (sender_t s, const msgbuf_t *msg);

int sender_run (sender_t s, int fd);

void sender_del (sender_t s);
\end{lstlisting}

\subsection{Receive from Peer}
\begin{lstlisting}
typedef enum {
    RECVER_MSG_READY,
    RECVER_EMPTY,
    RECVER_BUSY
} recver_state_t;
\end{lstlisting}

\subsubsection{Receiver - Interface}
\begin{lstlisting}
typedef struct recver * recver_t;

recver_t recver_new ();

recver_state_t recver_state (recver_t r);

void recver_reset (recver_t r);

const msgbuf_t * recver_read (recver_t r);

int recver_run (recver_t r, int fd);

void recver_del (recver_t r);
\end{lstlisting}

\subsection{Wait for Data}

\subsubsection{Wait for data - interface}
\begin{lstlisting}
\end{lstlisting}
