\section{Introduction}
\label{ch:intro}
Before getting lost in many technicalities, there have been reported a small
introduction to the GRAPES library. It is a generic environment for P2P
streaming. It is designed to be usable in different environments and
situations. Peer-to-peer (\textbf{P2P}) tecnologies are becoming increasingly
popular as a way to overcome the scalability problem limitations intrinsic in
traditional applications based on client/server paradigm. In particular there
is a great interest in P2P streaming applications because they have high
demands in term of bandwidth requirements and IP-level multicasting is not
supported on the internet.
More details could be found in \cite{disi10-038}.

The \textbf{GRAPES} library provides a net-helper for managing the
communication. At the beggining, it was using the UDP protocol on POSIX
systems. There were implemented just some functionalities which aim was to
provide communication without worring to much about reliability and
correctness: if the application needed such requisites, it had to warry of
them at application level. %(\textit{should we say this or is it offensive for the guys?! =P})

In this report we will explain the internals of the tcp-communication module.
In particular, we will focus on three main functions that are involved:
\begin{itemize}
  \item \textit{send_to_peer}:
  \item \textit{recv_from_peer}:
  \item \textit{wait4data}:
\end{itemize}
More details about \textbf{yada} can be found in \cite{yada1}, \cite{yada2},
\cite{yada3} and \cite{yada4}.
The first approach for studying the structure of these entities is by using
yada...
That is:
\begin{itemize}
\item a
\item b
\item c
\end{itemize}
