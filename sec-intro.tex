\section{Introduction}
\label{ch:intro}
Before getting lost in many technicalities, here has been reported a small
introduction to the GRAPES library. It is a generic environment for P2P
streaming and it is designed to be usable in different environments and
situations. More details could be found in \cite{disi10-038}.\\
The \textbf{GRAPES} \textit{(Generic Resource-Aware P2P Environment for Streaming)} library provides a tool,
called net-helper, for managing the communication. 
In particular, the main focus will be on the communication procedures:
\begin{itemize}
  \item \textbf{send\_to\_peer}: executes the sending of a messages. More details about
    this could be found in the further sections.
  \item \textbf{recv\_from\_peer}: executes the receiving of a message. More details about
    this could be found in the further sections.
  \item \textbf{wait4data}:  continuosly scans connected peers for receiving data.
    More details about this could be found in next section.
\end{itemize}
There have been tackled different problems that increase complexity on managing the communication in
\textbf{TCP} protocol respect to what was done in \textbf{UDP}.\\
On developing such procedures in \textbf{TCP}, the first problem encountered reguards connectivity.
\begin{itemize}
 \item On the one hand, the \textbf{UDP} protocol is not connection oriented. It uses the same file descriptor
       for communicating with all (connected) hosts.
 \item on the other hadnd, the \textbf{TCP} protocol needs to use multiple file descriptors, one for each
   connected peer and one for the server.
\end{itemize}
Considering the later point, there have been built a tool for managing such connections, called
\textit{Dictionary}.\\
Another not so trivial problem to handle is the simultaneous communication with different peers through different
file descriptors. This problem do not exist in \textbf{UDP} considering that there is a single file descriptor
for communicating with other hosts, whereas in \textbf{TCP} has been straighten out by using a specific
queue structure, called \textit{A-queue}.

\subsection{Dictionary}
This could be considered the key point for management of the connections. It is handled through a hash-table
which relies in an external library, \textbf{libdacav}.\\
The related data structure is declare as:
\begin{lstlisting}
struct dict {
    dhash_t *hash;
    size_t nelems;
};
\end{lstlisting}
(And some more yada to be added on the dictionary part...)

\subsection{A-queue}

\begin{lstlisting}
struct aqueue {
    dlist_t * queue;
    fd_set read;
    fd_set write;
    int maxfd;
    int n_active;
};
\end{lstlisting}
(And some more yada to be added on the aqueue part...)
