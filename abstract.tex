\begin{abstract}
In this report it is explained a module-extension for \textbf{GRAPES} library.
It consists of a practical implementation of a module that manages the
transmittion over TCP. In the previous version of \textbf{GRAPES} the
transmission was
executed over UDP. On the one hand, the advantages of TCP over UDP are quite
clear, as TCP guarantees that the sent data actually arrives, that it arrives
in order and that there are no duplicates, while UDP provides none of these
guarantees (only offers best effort policy). Unlike TCP, UDP does not provide
any flow and congestion control. On the other hand, UDP is more suitable in
applications and enviroments such peer-to-peer streaming, where thorughput
performance is the main request. It is faster because there is no
error-checking for packets, it does not order packets and there is no
tracking connections. It is a small transport layer designed on top of IP.
If any ordering is required, it has to be managed by the application layer.\\
Considering these aspects, there have been choose to develop a module that
consideres advantages from both protocols and takles some drawbacks derived
from them.

\end{abstract}
